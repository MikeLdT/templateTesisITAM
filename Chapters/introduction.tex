\chapter{Introducción}
\label{ch:intro}

%\begin{chapterquote}{Leslie Lamport}
%	Formal mathematics is nature's way of letting you know how sloppy
%your mathematics is.
%\end{chapterquote}

%My objective is to analyze the impact in 2019 of the minimum wage increase in the Zona Libre de la Frontera Norte (ZLFN) with respect to the rest of the country had in housing credits offered by INFONAVIT, housing prices and private mortgages. Using data from IMSS and INFONAVIT combined with a dif-in-dif strategy, I measure the impact of such policy in wages reported to IMSS, and also in the number and quantity of INFONAVIT's housing credits. Following the same methodology, from SHF and CNBV data, I estimate the impact on housing prices and private mortgages.

\noindent Los aumentos al salario han captado gran atención de la literatura que busca medir su impacto. Además de los canales tradicionales (por ejemplo consumo, desempleo e ingreso), incluso los precios minoristas pueden ser afectados por subir el salario mínimo \citep{leung_2020}. Hay argumentos para pensar que puede haber transisión al mercado inmobiliario \citep{macurdy_2015,yamagishi_2018,agarwal_ambrose_diop_2019}, lo que podría afectar los precios de los hogares. En particular, en México el salario mínimo está ligado al salario reportado ante las instituciones de seguridad social, que a su ves está ligado al crédito hipotecario otorgado por el gobierno. En ese sentido, resulta interesante preguntarse el efecto de ese mecanismo en el mercado inmbiliario. 

Mi objetivo es analizar el impacto que tuvo en 2019 el aumento diferenciado del salario mínimo en la Zona Libre de la Frontera Norte (ZLFN) con respecto al resto del país sobre los créditos hipotecarios ofrecidos por el Instituto del Fondo Nacional de la Vivienda para los Trabajadores (INFONAVIT), los precios de la vivienda y los créditos privados. En concreto, utilizando datos del Instituto Mexicano del Seguro Social (IMSS) y del INFONAVIT, así como una estrategia de diferencias en diferencias, mido el impacto de esta política sobre los salarios de cotización ante el IMSS, así como en el monto y número de créditos otorgados por el INFONAVIT. De la misma manera, a partir de datos de la Sociedad Hipotecaria Federal (SHF) y la Comisión Nacional Bancaria y de Valores (CNBV) mido el impacto en los precios de los hogares y en los créditos hipotecarios privados.

%My principal hypothesis is that the minimum wage increase could have an impact on the housing market via two mechanisms. First, and clearly,  worker's purchase power is enhanced. Second, salaries reported to IMSS increase, and there is substantial evidence that bosses misreport workers' income \citep{kumler_verhoogen_frias_2020}. I do not expect that private mortgages are affected since workers affected by this policy usually borrow from INFONAVIT.

Mi hipótesis principal es que el aumento al salario mínimo puede tener un impacto en la cantidad de créditos hipotecarios otorgados a través de dos mecanismos. El primero y más evidente, es el aumento en el ingreso de los trabajadores y por lo tanto, en su poder adquisitivo y capacidad de pago. El segundo mecanismo es precisamente a través del aumento en el porcentaje de los ingresos laborales totales que los empleadores reportan como salarios al IMSS, pues existe evidencia que sugiere que un monto importante de los ingresos laborales no es reportado al IMSS \citep{kumler_verhoogen_frias_2020}. No es de esperar que caeteris paribus el aumento afecte los créditos privados, ya que atienden a otro sector de la distribución de salarios. 

%I present three sets of results. First, I verify that salaries reported to IMSS increase and that employment is unchanged. Second, given this increase, benefiting especially low earners, I document an increase in lending by INFONAVIT in states affected mostly by this policy. Finally, I find that neither private lending nor housing prices changed as a result of this policy.

Presento tres conjuntos de resultados principales. Primero, verifico que ante un aumento al salario mínimo suben los salarios de cotización ante el IMSS y además, se mantiene sin cambio el nivel de empleo. Después, dado este aumento en el ingreso laboral de los trabajadores, sobre todo aquellos de menores ingresos, documento una expansión de créditos otorgados por el INFONAVIT en los estados del país en donde el aumento al salario mínimo en la ZLFN afectó a la población trabajadora. Finalmente, no hay cambio en el crédito privado, y el aumento de crédito INFONAVIT no se traduce por sí solo en algún cambio en los precios de vivienda. Podemos inferir entonces un aumento en el bienestar de quienes están en la cola izquierda de la distribución de salario, que ahora tienen más crédito disponible a los mismos precios, por lo que tienen acceso a mayores opciones de vivienda.

%We can infer there is an increase in consumers' welfare, specifically for those at the left of the wages distribution. Now, less-earning people have access to more credit and prices remain unchanged, thus can buy improved-quality houses.

