\chapter{Revisión de Literatura}
\label{ch:relatedwork}

%This paper is related with empirical studies around two topics: measuring minimum wage's impact and 

\section{Salario mínimo}
%There exist wide variety of literature around the impact of increasing the minimum wage for developing countries. Estimates of the elasticity of employment with respect to minimum wage are generally small and heterogeneous (Neumark \& Munguia Corella, 2019). The evidence suggests that for the mexican context, especially when employers have monopsonic power, the impact of the minimum wage in employment can be small or insignificant. On the other hand, increasing the minimum wage can impact income and the percentage of work income that is reported to IMSS (Munguia Corella, 2020).

\noindent Existe una amplia literatura sobre el impacto del aumento al salario mínimo en países en vías de desarrollo. Estimaciones de la elasticidad del empleo respecto al salario mínimo son generalmente pequeñas y heterogéneas \citep{neumark_2019}. La evidencia entonces sugiere que para contextos como el mexicano, sobre todo en situaciones en que las empresas tienen poder monopsónico, el impacto del salario mínimo en el empleo puede ser muy pequeño o nulo. No obstante, el aumento del salario mínimo sí impacta los ingresos laborales y el porcentaje de los ingresos laborales que se reportan al IMSS \citep{munguia_2020}. 


%In respect to the Mexican case, the minimum wage increase in ZLFN with respect to the rest of the country in 2019 came with an important number of research papers that attempt to measure its impact on employment and salaries. On one hand, reports by Banco de México and some other researchers find a negative effect in employment \citep{banxico2020} and prices \citep{banxico2020, calderon2020}. On the other hand, other authors find no such impact in employment \citep{campos-vazquez_delgado_rodas_2020} nor in prices \citep{campos-vazquez_esquivel_2020}, but they recognize that it could be caused in part by VAT's decrease that came with the minimum wage increase.
 
 

Respecto al caso mexicano, el aumento del salario mínimo diferenciado en la ZLFN con respecto al resto del país en 2019 vino acompañado de un número importante de artículos de investigación que buscan medir su impacto en el empleo y los salarios. Por un lado, los estudios realizados por Banco de México y algunos otros investigadores señalan que sí hubo efectos negativos en la generación de empleo \citep{banxico2020} y en los precios \citep{banxico2020, calderon2020}. Sin embargo, otros autores no encuentran efectos significativos en el empleo \citep{campos-vazquez_delgado_rodas_2020} ni en los precios al consumidor \citep{campos-vazquez_esquivel_2020}, aunque reconocen que puede deberse en parte a que el aumento al salario mínimo en 2018 fue acompañado de una disminución del Impuesto al Valor Agregado (IVA). La propia Comisión Nacional de los Salarios Mínimos, en su reporte del 2019, concluye que el aumento salarial no provocó efectos negativos en la generación de empleo.

%With respect to mortgages, there exists evidence that exogenous shocks to credit supply can affect housing prices \citep{imbs_favara_2010}. Since house prices can affect a households' wealth, movements in the housing market can modify consumption decisions. Also, a property's values influence firms' and families' location decisions. In a macro setting, housing prices have played an important role in recent business cycles. Then, public policies aimed at increasing credit supply must take into account the aforementioned effects.

A pesar de estas diferencias, los artículos antes mencionados coinciden en que, para el caso de México, los efectos sobre el empleo parecen no ser significativos. Además, la mayoría de los estudios coinciden en que esta política sí tuvo como consecuencia un aumento en los salarios de cotización ante el IMSS. 


\section{Vivienda}

\noindent La construcción de desarrollos habitacionales de bajo costo fue un pilar fundamental de la política pública de países desarrollados durante el s. XX \citep{monkkonen_2011}. Sin embargo, a partir del s. XXI, el paradigma cambió hacia una política habitacional que dejó en manos del mercado inmobiliario la construcción de nuevos asentamientos. La política pública tomó un enfoque de oferta, otorgando subsidios o beneficios fiscales para que las empresas inmobiliarias construyeran viviendas de bajo costo. Sin embargo, estas viviendas se construían principalmente en la periferia de las ciudades, lejos del transporte público y otros servicios; ello provocaba  segregación y pérdidas en bienestar para la población de menores ingresos. La literatura indica que vivir en asentamientos lejos del centro económico de las ciudades en países en desarrollo puede impactar la experiencia laboral de los jóvenes \citep{franklin_2018,kain_1992} y la mayoría de los propietarios prefieren abandonarlos \citep{barnhardt_2017}. 

 
En contraste, soluciones por el lado de la demanda pueden mejorar la movilidad intergeneracional de las personas con menores ingresos. Por ejemplo, para el contexto de Estados Unidos, individuos que se mudaron a zonas de baja pobreza antes de cumplir trece años tienen, significativamente, mayor escolaridad e ingreso con respecto a sus congéneres que no se mudaron \citep{chetty_hendren_katz_2016,chetty_hendren_2018}. Adicionalmente para el contexto de familias estadounidenses viviendo en barrios urbanos marginados, existe evidencia de beneficios importantes derivados de mudarse fuera de zonas con altos niveles de pobreza, tales como menores tasas de obesidad, diabetes y, en general, menores afectaciones psicológicas y menor prevalencia de depresión y ansiedad \citep{kling_liebeman_katz_2007}. Por tanto, fomentar que trabajadores de bajos ingresos puedan acceder a vivienda en mejores vecindarios puede expandir sus posibilidades de desarrollo.


En cuanto al crédito hipotecario, existe evidencia de que shocks exógenos a la oferta de crédito pueden afectar los precios de las viviendas \citep{imbs_favara_2010}. Saber si movimientos en la oferta de crédito hipotecario afectan directamente a los precios del mercado inmobiliario es crucial para políticas públicas. El valor de las propiedades inmobiliarias afecta el ingreso de los hogares y por tanto puede impactar sus decisiones de consumo. Además, determina la ubicación de los hogares y de las firmas. En un nivel macro, los precios de los hogares han jugado un papel importante en los ciclos de negocios recientes.

%Since houses are usually used as collateral for mortgages, there is endogeneity between housing credit supply and housing prices. A way to explore the impact of credit booms is to exploit politically motivated policies that impact the credit supply but do not correlate with housing prices. In my case, I propose the minimum wage increase in ZLFN for such analysis.

En cuanto al crédito hipotecario, ya que regularmente el colateral es un propiedad es difícil aislar el efecto de los precios de vivienda en la oferta de hipotecas. Es decir, la relación entre el precio de las viviendas y el crédito hipotecario es endógena. Un camino para identificar el impacto de movimientos en la oferta crediticia en los precios de vivienda es explotar el impacto de decisiones políticas que mueven la oferta de crédito pero no los precios \citep{imbs_favara_2010}. En ese sentido, a mi me interesa explotar el aumento del salario mínimo en la ZLFN.

%Also, recently there is renewed interest around the progressivity of increasing the minimum wage. There is evidence that the channels by which increasing the minimum wage affects consumers are more diverse than was thought of. Apart from the traditional channels (e.g. unemployment, income, consumption), even retail prices can be negatively affected by increasing the minimum wage \citep{leung_2020}. There are arguments to think that spillover to the housing market might happen, which in turn makes housing rents go up \citep{macurdy_2015,yamagishi_2018,agarwal_ambrose_diop_2019}.
%Since low earners tend to rent rather than own a house, this might influence welfare negatively.

Además, en un contexto reciente donde se cuestiona que subir el salario mínimo sea una medida progresiva, hay razones para pensar en el mercado inmobiliario como un canal por el cual se diseminan los efectos del aumento al salario mínimo. Existe evidencia de sobra sobre el resultado de subir el salario mínimo en el ingreso, el consumo y el desempleo por citar ejemplos, e incluso sobre los precios minoristas \citep{leung_2020} . Recientemente se ha encontrado que las subidas de salario mínimo terminan causando que suban las rentas \citep{macurdy_2015,yamagishi_2018,agarwal_ambrose_diop_2019} y dado que quienes menos ganan tienden a rentar en vez de poseer vivienda, no podemos asegurar que su bienestar mejore.