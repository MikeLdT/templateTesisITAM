\chapter{Discusión}
\label{ch:conclusions}

%\begin{chapterquote}{Leslie Lamport}
%	Formal mathematics is nature's way of letting you know how sloppy
%your mathematics is.
%\end{chapterquote}

%Yo empezaría por resumir lo que hiciste:
%Este trabajo responde a la pregunta del impacto de X en Y. Siguiendo el método Z, encontré W. 
%Después diría posibles interpretaciones o implicaciones de política pública. ¿Por qué tiene valor lo que hiciste?
%Por último, recalcaría limitaciones de tu análisis: datos y metodología.

\noindent Este trabajo presenta una primera aproximación al impacto del aumento del salario mínimo en el mercado inmobiliario, en particular al acceso a vivienda para quienes perciben menores ingresos. A partir del método de diferencias en diferencias logro aislar el efecto promedio de subir el salario mínimo en el empleo y salario para trabajadores registrados ante el IMSS, el efecto en los créditos INFONAVIT, en las hipotecas privadas y en los precios de vivienda. 

Encuentro que aumentar el salario mínimo no afectó ni al empleo, ni a los créditos privados ni a los precios de vivienda. Sin embargo, sí causó que aumentaran tanto el salario reportado ante el IMSS (en mayor medida para quienes menor salario tienen) como el crédito INFONAVIT. Cabe resaltar que no puedo hablar de un aumento en el ingreso laboral, ya que no tengo evidencia de que el aumento al salario no se dio a costa de una reducción en prestaciones.

La principal implicación de política pública de mi análisis es que conduce a pensar que las medidas encaminadas a aumentar el porcentaje del salario que es reportado por los patrones trae mejoras al bienestar de quienes menos ganan. Al cotizar con mayores salarios ante el IMSS pueden obtener más crédito INFONAVIT y enfrentar en promedio los mismos precios de vivienda. Entonces, pueden adquirir mejores hogares, posiblemente más cerca de los centros de trabajo o con mejores servicios.



Todos los datos que utilicé son de acceso público, disponibles en las bases de datos de las distintas instituciones (IMSS,INFONAVIT,SHF,CNBV) que cito. En ese sentido, mis conclusiones están limitadas por los datos que están disponibles al público. Para llegar a mejores conclusiones sería necesario contar con datos más desagregados. En concreto, dado que cada sector de la distribución de salario demanda distintas opciones de vivienda, un análisis de heterogeneidad sería altamente valioso. 

Como los datos para el precio proviene de un índice agregado, este análisis carece de herramientas para determinar si quizás se suscitaron  movimientos en los precios de las casas demandadas por quienes ganan menos. Se podría dilucidar este cuestionamiento   si los microdatos que se utilizaron para construir el índice fuesen de acceso público.

Quizá mis conclusiones serían más profundas si pudiera analizar por banco, ya que hay bancos que atienden a un sector particular de la demanda. Sería interesante ver un análisis en crédito al consumo, donde puede ser que las viviendas recién adquiridas se utilicen como colateral para financiar otro tipo de inversiones.

Más importante aún, puede ser que el efecto de aumentar el salario mínimo se diluya por otros canales que no observamos. Por ejemplo, quienes ganan menos salarios regularmente rentan en lugar de comprar. Además, quienes compran adquieren viviendas con características distintas a quienes se ubican en otros puntos de la distribución de salarios. Entonces, mi análisis se encuentra limitado al no tener datos sobre las rentas de vivienda.

Finalmente, es imposible distinguir si un movimiento en precios agregados por aumentar el salario mínimo proviene del puro efecto ingreso o del canal de crédito. En caso de disponer de microdatos, podría seguir un análisis como el propuesto por \citep{diamond_2016} para capturar qué tanto valoran los individuos vivir en una región con un salario mínimo comparativamente más alto.
